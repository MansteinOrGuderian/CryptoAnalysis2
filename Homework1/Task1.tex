\chapter{}
\label{chap:1}

\section{Умова}
Побудувати граф скінченного автомата та визначити, чи є цей автомат оборотним за Гаффманом, якщо
\begin{enumerate}
    \item $X = S = Y = \left\{0, 1\right\}$, $h(s, x) = s \cdot x \oplus s \oplus 1$, $f(s, x) = s \cdot x$;
    \item $X = Y = \left\{0, 1\right\}, S = \left\{0, 1, 2, 3\right\}$, $f(s, x) = 0 \, \forall s \in S, \, x \in X$, окрім $f(3, 1) = 1$, \\
          $h(0, 0) = h(2, 0) = 0$, $h(0, 1) = h(2, 1) = 1$, $h(1, 0) = h(3, 0) = 2$, $h(1, 1) = h(3, 1) = 3$.
\end{enumerate}

\section{Розв'язання}

Простими словами (з лекції) оборотність автомата за Гаффманом визначалася так: коли за будь-якою вихідною послідовністю
та парою станів (початковим і фінальним) можна однозначно відновити відповідну їм вхідну послідовність.

Або іншими словами: автомат називатиметься \textbf{оборотним за Гаффманом}, якщо для кожного стану $s \in S$ функція виходу
$f(s, \cdot): X \to Y$ є ін'єктивною, тобто:
\begin{equation*}
    \forall s \in S, \, \forall x_1, x_2 \in X: \quad f(s, x_1) = f(s, x_2) \Rightarrow x_1 = x_2
\end{equation*}

\subsection*{Пункт 1}

У нас задано вхідний алфавіт, множину станів, множина виходів (вихідний алфавіт), функцію переходів та функцію оновлення стану \\
$X = S = Y = \{0, 1\}$, \quad $h(s, x) = s \cdot x \oplus s \oplus 1$, \quad $f(s, x) = s \cdot x$.

\subsubsection*{Таблиця переходів \textit{h} та виходів \textit{f}}

\begin{center}
    \begin{tabular}{cc|c|c}
        \toprule
        $s$ & $x$ & $h(s,x)$                          & $f(s,x)$        \\
        \midrule
        $0$ & $0$ & $0 \cdot 0 \oplus 0 \oplus 1 = 1$ & $0 \cdot 0 = 0$ \\
        $0$ & $1$ & $0 \cdot 1 \oplus 0 \oplus 1 = 1$ & $0 \cdot 1 = 0$ \\
        $1$ & $0$ & $1 \cdot 0 \oplus 1 \oplus 1 = 0$ & $1 \cdot 0 = 0$ \\
        $1$ & $1$ & $1 \cdot 1 \oplus 1 \oplus 1 = 1$ & $1 \cdot 1 = 1$ \\
        \bottomrule
    \end{tabular}
\end{center}

Можна зобразити автомат графічно:

\begin{center}
    \begin{tikzpicture}[
            ->,
            >=Stealth,
            node distance=4cm,
            every state/.style={thick, minimum size=1cm},
            every edge/.style={draw, thick}
        ]
        \node[state] (s0) {$0$};
        \node[state, right=of s0] (s1) {$1$};

        % Переходи зі стану 0
        \path (s0) edge[bend left=10] node[above] {$0/0$} (s1);
        \path (s0) edge[bend left=40] node[above] {$1/0$} (s1);

        % Переходи зі стану 1
        \path (s1) edge[bend left=20] node[below] {$0/0$} (s0);
        \path (s1) edge[loop right] node[right] {$1/1$} (s1);
    \end{tikzpicture}
\end{center}

\subsubsection*{Перевіримо оборотність за Гаффманом}

Маємо перевірити ін'єктивність $f(s, \cdot)$ для кожного стану:

\begin{itemize}
    \item \textbf{Стан $s = 0$:} $f(0, 0) = 0$ і $f(0, 1) = 0$.
        Різні входи ($x = 0$ і $x = 1$) дають однаковий вихід --- не ін'єктивно.
    \item \textbf{Стан $s = 1$:} $f(1, 0) = 0$ і $f(1, 1) = 1$.
        Різні входи $x$ дають різні виходи --- ін'єктивно.
\end{itemize}

\noindent Отже, автомат \textbf{не є оборотним за Гаффманом}, оскільки для стану $s = 0$ функція виходу не є ін'єктивною.

%==============================================================================
\subsection*{Пункт 2}

Задано $X = Y = \{0, 1\}$, $S = \{0, 1, 2, 3\}$.

Функція переходів (всі можливі випадки перебрані):
\begin{equation*}
    \begin{aligned}
        h(0, 0) & = 0, & h(0, 1) & = 1, \\
        h(1, 0) & = 2, & h(1, 1) & = 3, \\
        h(2, 0) & = 0, & h(2, 1) & = 1, \\
        h(3, 0) & = 2, & h(3, 1) & = 3.
    \end{aligned}
\end{equation*}

Функція виходу: $f(s, x) = 0$ для всіх $(s, x)$, окрім $f(3, 1) = 1$.

\subsubsection*{Згрупуємо все в таблицю:}

\begin{center}
    \begin{tabular}{cc|cc}
        \toprule
        $s$ & $x$ & $h(s,x)$ & $f(s,x)$ \\
        \midrule
        $0$ & $0$ & $0$      & $0$      \\
        $0$ & $1$ & $1$      & $0$      \\
        $1$ & $0$ & $2$      & $0$      \\
        $1$ & $1$ & $3$      & $0$      \\
        $2$ & $0$ & $0$      & $0$      \\
        $2$ & $1$ & $1$      & $0$      \\
        $3$ & $0$ & $2$      & $0$      \\
        $3$ & $1$ & $3$      & $1$      \\
        \bottomrule
    \end{tabular}
\end{center}

\noindent Так само зобразимо графічно:

\begin{center}
    \begin{tikzpicture}[
            ->,
            >=Stealth,
            node distance=3cm,
            every state/.style={thick, minimum size=1cm},
            every edge/.style={draw, thick}
        ]
        \node[state] (s0) {$0$};
        \node[state, right=of s0] (s1) {$1$};
        \node[state, below=of s0] (s2) {$2$};
        \node[state, right=of s2] (s3) {$3$};

        % Переходи зі стану 0
        \path (s0) edge[loop above] node[above] {$0/0$} (s0);
        \path (s0) edge node[above] {$1/0$} (s1);

        % Переходи зі стану 1
        \path (s1) edge[bend right=10] node[above left] {$0/0$} (s2);
        \path (s1) edge node[right] {$1/0$} (s3);

        % Переходи зі стану 2
        \path (s2) edge node[left] {$0/0$} (s0);
        \path (s2) edge[bend right=10] node[below right] {$1/0$} (s1);

        % Переходи зі стану 3
        \path (s3) edge node[below] {$0/0$} (s2);
        \path (s3) edge[loop right] node[right] {$1/1$} (s3);
    \end{tikzpicture}
\end{center}

\subsubsection*{Перевіримо оборотність за Гаффманом}

Перевіряємо ін'єктивність $f(s, \cdot)$ для кожного стану:

\begin{itemize}
    \item \textbf{Стан $s = 0$:} $f(0, 0) = 0$ і $f(0, 1) = 0$ --- \textbf{не ін'єктивно}.
    \item \textbf{Стан $s = 1$:} $f(1, 0) = 0$ і $f(1, 1) = 0$ --- \textbf{не ін'єктивно}.
    \item \textbf{Стан $s = 2$:} $f(2, 0) = 0$ і $f(2, 1) = 0$ --- \textbf{не ін'єктивно}.
    \item \textbf{Стан $s = 3$:} $f(3, 0) = 0$ і $f(3, 1) = 1$ --- ін'єктивно.
\end{itemize}

\noindent Отже, автомат \textbf{не є оборотним за Гаффманом}, оскільки для станів $s \in \{0, 1, 2\}$ функція виходу не є ін'єктивною.
