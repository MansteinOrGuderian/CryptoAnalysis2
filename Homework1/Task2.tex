\chapter{}
\label{chap:2}

\section{Умова}
Нехай $\Gamma$ -- генератор гами з множиною станів $V_{n}$ та вихідним алфавітом $V_{2}$, який виробляє за
початковим станом $s_{0}$ вихідну послідованість $\Gamma_{L} (s_{0})$ довжини $L$. Покажіть, що існує статистичний
критерій, який дозволяє відрізнити цю послідовність, отриману за випадковим рівноймовірним початковим станом, вiд
суто випадкової двійкової послідовності довжини $L$ iз середньою ймовірністю помилки $p_{e}$, використовуючи $T$
двійкових операцій, якщо $\Gamma_{2N} : s_{0} → (s_{0}, s_{0})$, $p_{e} = 2^{-N-1}$, $T = N$.

\section{Розв'язання}

З умови можна зробити висновок, що послідовність $\Gamma_{2N} : s_0 \to (s_0, s_0)$ довжини $2N$ має вигляд:
\begin{equation*}
    \Gamma_{2N}(s_0) = (\gamma_1, \gamma_2, \ldots, \gamma_N, \gamma_1, \gamma_2, \ldots, \gamma_N)
\end{equation*}

\noindent Тобто \textbf{перші $N$ біт повторюються в наступних $N$ бітах}.

\subsection*{Побудова статистичного критерію}

\textbf{Критерій:} Для послідовності $x = (x_1, x_2, \ldots, x_{2N})$ перевіряємо рівність:

\begin{equation*}
    D(X) = \sum\limits_{i=1}^{N} x_{i} \oplus x_{N+i} =
    \begin{cases*}
        0, \, \Leftrightarrow x_{i} = x_{N+i}, \quad \forall \, i, \, i = \overline{1, N} \quad \Rightarrow x \in \Gamma_{L} \\
        \neq 0, \, - \, random \, sequence
    \end{cases*}
\end{equation*}

Висунемо такі гіпотези:
\begin{itemize}
    \item $H_0$: послідовність від генератора $\Gamma$
    \item $H_1$: суто випадкова, рівноймовірна послідовність
\end{itemize}

\textbf{Помилка I роду} (хибне відхилення $H_0$):
\begin{equation*}
    \alpha = P \left(H_{1} \mid H_{0}\right) = 0
\end{equation*}
Нуль, бо генератор \textbf{завжди} видає $(x_1, \ldots, x_N) = (x_{N+1}, \ldots, x_{2N})$ за умовою.

\textbf{Помилка II роду} (хибне прийняття $H_0$):
\begin{equation*}
    \beta = P \left(H_0 \mid H_1\right) = P(x_i = x_{N+i}, \forall i \mid \text{випадкова})
\end{equation*}

Для випадкової послідовності біти є незалежними одне від одного, тому:
\begin{equation*}
    \beta = \prod_{i=1}^{N} P \left(x_i = x_{N+i}\right) = \prod_{i=1}^{N} \frac{1}{2} = 2^{-N}
\end{equation*}

\noindent Середня ймовірність помилки обчислюється за Байєсом:

\begin{equation*}
    p_{e} = \frac{1}{2} \left(\alpha + \beta\right) = \frac{1}{2} \cdot 0 + \frac{1}{2} \cdot 2^{-N} = 2^{-N-1}
\end{equation*}

Обчислювальна складність цього критерію:

$N$ XOR-ів: $x_i \oplus x_{N+i}$ для $i = 1, \ldots, N$, тобто загальна кількість двійкових операцій $T = N$

У висновку можна сказати, що критерій експлуатує детерміновану структурну слабкість генератора -- періодичність з 
періодом $N$, яка неможлива для справді випадкової послідовності з ймовірністю $1 - 2^{-N}$.
